\documentclass{article}

\usepackage{syntax}
\usepackage[T1]{fontenc}			% https://tex.stackexchange.com/questions/2369/why-do-the-less-than-symbol-and-the-greater-than-symbol-appear-wrong-as

\usepackage[english]{babel}
\usepackage[utf8]{inputenc}

\usepackage{pdfpages}

\usepackage{graphicx}
\graphicspath{ {images/} }




\begin{document}

	\setlength{\grammarparsep}{5pt plus 1pt minus 1pt} % increase separation between rules
%	\setlength{\grammarindent}{12em} % increase separation between LHS/RHS 
	\setlength{\grammarindent}{5cm} 




	\begin{titlepage}
		\vspace*{\stretch{1.0}}
		\begin{center}
				\Large\textsc{CSc 600-01 (Section 1)}
				
				\Large\textbf{Homework 1 - Syntax}\\
				
				\Large\textit{prepared by Ilya Kopyl}
		\end{center}	
		\vspace*{\stretch{2.0}}
	\end{titlepage}


%	\title{\textsc{CSc 600 Homework 1 - Syntax}}	
%	\maketitle
	
		Homework is prepared by: Ilya Kopyl.

		It is formatted in LaTeX, using TeXShop editor (under GNU GPL license).
		
		Syntax diagrams are created in LucidChart online editor (lucidchart.com).


	\rmfamily\




	\paragraph{1. Using BNF write the syntax definitions of the following objects:}\
	\rmfamily\\\
	
		a) Natural number (1, 2, 3, ...). The answer:
			
	\ttfamily
	\begin{grammar}
	
		<natural number> ::= <non-zero digit> | <natural number> <digit>

		<digit> ::= 0 | <non-zero digit>

		<non-zero digit> ::= 1 | 2 | 3 | 4 | 5 | 6 | 7 | 8 | 9
		
	\end{grammar}
	
	

	
	\paragraph{}
	\rmfamily
	
		b) Unsigned integer (0, 1, 2, 3, ...). The answer:
			
	\ttfamily
	\begin{grammar}
		
		<unsigned integer> ::= 0 | <natural number>
		
		<natural number> ::= <non-zero digit> | <natural number> <digit>

		<digit> ::= 0 | <non-zero digit>

		<non-zero digit> ::= 1 | 2 | 3 | 4 | 5 | 6 | 7 | 8 | 9
		
	\end{grammar}




	\paragraph{}
	\rmfamily
	
		c) Integer (..., -2, -1, 0, 1, 2, ...). The answer:
			
	\ttfamily
	\begin{grammar}
		
		<integer> ::= <sign> <unsigned integer>
		
		<sign> ::= + | - | <empty>
		
		<empty> ::= \
		
		<unsigned integer> ::= 0 | <natural number>
		
		<natural number> ::= <non-zero digit> | <natural number> <digit>

		<digit> ::= 0 | <non-zero digit>
		
		<non-zero digit> ::= 1 | 2 | 3 | 4 | 5 | 6 | 7 | 8 | 9

		
	\end{grammar}




	\paragraph{}
	\rmfamily
	
		d) Odd number (..., -3, -1, 1, 3, ..., 101, ..., 2047, ...). The answer:
			
	\ttfamily
	\begin{grammar}
	
		<odd number> ::= <sign> <unsigned odd number>
		
		<sign> ::= + | - | <empty>
		
		<empty> ::= \
		
		<unsigned odd number> ::= <odd digit> | <natural number> <odd digit>
		
		<natural number> ::= <non-zero digit> | <natural number> <digit>
		
		<digit> ::= 0 | <non-zero digit>
		
		<non-zero digit> ::= 2 | 4 | 6 | 8 | <odd digit>
		
		<odd digit> ::= 1 | 3 | 5 | 7 | 9
		
	\end{grammar}
	
	
	
	
	\paragraph{}
	\rmfamily
	
		e) Even number (..., -4, -2, 0, 2, 4, ..., 332, ..., 1022, ...). The answer:
		
	\ttfamily
	\begin{grammar}
	
		<even number> ::= <sign> <unsigned even number>
	
		<sign> ::= + | - | <empty>
	
		<empty> ::= \
	
		<unsigned even number> ::= <even digit> | <natural number> <even digit>
	
		<natural number> ::= <non-zero digit> | <natural number> <digit>
	
		<digit> ::= 0 | <non-zero digit>
	
		<non-zero digit> ::= 1 | 2 | 3 | 4 | 5 | 6 | 7 | 8 | 9
	
		<even digit> ::= 0 | 2 | 4 | 6 | 8
	
			
	\end{grammar}
	
	
	
	
	\paragraph{}
	\rmfamily
	
		f) Integer divisible by five (..., -10, 5, 0, 5, 10, ...). The answer:
		
	\ttfamily
	\begin{grammar}
	
		<integer div-by-5> ::= <sign> <unsigned int div-by-5>
		
		<sign> ::= + | - | <empty>
		
		<empty> ::= \
		
		<unsigned int div-by-5> ::= <div-by-5 suffix> | <natural number> <div-by-5 suffix>
		
		<natural number> ::= <non-zero digit> | <natural number> <digit>
		
		<div-by-5 suffix> ::= 0 | 5
		
		<digit> ::= 0 | <non-zero digit>
		
		<non-zero digit> ::= 1 | 2 | 3 | 4 | 5 | 6 | 7 | 8 | 9
		
	\end{grammar}
	
	
	
	
	
	\paragraph{2. Show syntax diagrams for questions (a), ..., (f) of problem 1.}

			
%		\includegraphics{diagrams}
		\includegraphics[width=\textwidth]{leadingzeroes3}
		\rmfamily
		Example of syntax diagrams for integers with no support of leading zeroes.
		\includegraphics[width=\textwidth-28pt]{noleadingzeroescrop2}
 %		\includepdf{shortdiagram.png}
		
			
	\ttfamily



\end{document}