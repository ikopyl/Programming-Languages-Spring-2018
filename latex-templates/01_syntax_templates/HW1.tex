\documentclass{article}

\usepackage{syntax}
\usepackage[T1]{fontenc}			% https://tex.stackexchange.com/questions/2369/why-do-the-less-than-symbol-and-the-greater-than-symbol-appear-wrong-as

\usepackage{courier} % tt

\usepackage[english]{babel}
\usepackage[utf8]{inputenc}

\usepackage{pdfpages}

\usepackage{graphicx}
\graphicspath{ {images/} }

\usepackage{microtype}
\DisableLigatures[<]{encoding = T1}



\usepackage{listings}
\lstset{basicstyle=\ttfamily}



\begin{document}

	\setlength{\grammarparsep}{5pt plus 1pt minus 1pt} % increase separation between rules
%	\setlength{\grammarindent}{12em} % increase separation between LHS/RHS 
	\setlength{\grammarindent}{13em} % increase separation between LHS/RHS 
%	\setlength{\grammarindent}{5cm} 




	\begin{titlepage}
		\vspace*{\stretch{1.0}}
		\begin{center}
				\Large\textsc{CSc 600-01 (Section 1)}
				
				\Large\textbf{Homework 1 - Syntax}\\

				\Large\textit{prepared by Ilya Kopyl}
				
		\end{center}	
		\vspace*{\stretch{2.0}}
	\end{titlepage}


	\title{\textsc{CSc 600 Homework 1 - Syntax}}	
	\maketitle
	
		\textit{Homework is prepared by: Ilya Kopyl.}

		\textit{It is formatted in LaTeX, using TeXShop editor (under GNU GPL license).}
		
		\textit{Syntax diagrams are created in LucidChart online editor (lucidchart.com).}


	\rmfamily\




	\paragraph{1. Using BNF write the syntax definitions of the following objects:}\
	\rmfamily\\\
	
		a) Natural number (1, 2, 3, ...). The answer:
			
	\ttfamily
	\begin{grammar}
	
		<natural number> ::= <non-zero digit> | <natural number> <digit>

		<digit> ::= 0 | <non-zero digit>

		<non-zero digit> ::= 1 | 2 | 3 | 4 | 5 | 6 | 7 | 8 | 9
		
	\end{grammar}
	
	

	
	\paragraph{}
	\rmfamily\
	
		b) Unsigned integer (0, 1, 2, 3, ...). The answer:
		
	\ttfamily
	\begin{grammar}
		
		<unsigned integer> ::= <digit> | <unsigned integer> <digit>

		<digit> ::= 0 | 1 | 2 | 3 | 4 | 5 | 6 | 7 | 8 | 9
		
	\end{grammar}

		\rmfamily\
	
		Example of BNF definition of unsigned integer in languages that do not support leading zeroes (e.g. Python):
			
	\ttfamily
	\begin{grammar}
		
		<unsigned integer> ::= 0 | <natural number>
		
		<natural number> ::= <non-zero digit> | <natural number> <digit>

		<digit> ::= 0 | <non-zero digit>

		<non-zero digit> ::= 1 | 2 | 3 | 4 | 5 | 6 | 7 | 8 | 9
		
	\end{grammar}




	\paragraph{}
	\rmfamily
	
		c) Integer (..., -2, -1, 0, 1, 2, ...). The answer:
			
	\ttfamily
	\begin{grammar}
	
		<integer> ::= <sign> <unsigned integer>
		
		<sign> ::= + | - | <empty>
		
		<empty> ::= \
		
		<unsigned integer> ::= <digit> | <unsigned integer> <digit>

		<digit> ::= 0 | 1 | 2 | 3 | 4 | 5 | 6 | 7 | 8 | 9

	\end{grammar}

		\rmfamily\
	
		Example of BNF definition of an integer in languages that do not support leading zeroes (e.g. Python):
			
	\ttfamily
	\begin{grammar}
		
		<integer> ::= <sign> <unsigned integer>
		
		<sign> ::= + | - | <empty>
		
		<empty> ::= \
		
		<unsigned integer> ::= 0 | <natural number>
		
		<natural number> ::= <non-zero digit> | <natural number> <digit>

		<digit> ::= 0 | <non-zero digit>
		
		<non-zero digit> ::= 1 | 2 | 3 | 4 | 5 | 6 | 7 | 8 | 9
		
	\end{grammar}



	\paragraph{}
	\rmfamily\
	
		d) Odd number (..., -3, -1, 1, 3, ..., 101, ..., 2047, ...). The answer:
			
	\ttfamily
	\begin{grammar}
	
		<odd number> ::= <sign> <unsigned odd number>
		
		<sign> ::= + | - | <empty>
		
		<empty> ::= \
		
		<unsigned odd number> ::= <odd digit> | <unsigned integer> <odd digit>
		
		<unsigned integer> ::= <digit> | <unsigned integer> <digit>

		<digit> ::= 0 | 1 | 2 | 3 | 4 | 5 | 6 | 7 | 8 | 9

		
	\end{grammar}

		\paragraph{}	
		\rmfamily\
		
		Example of BNF definition of an odd number in languages that do not support leading zeroes (e.g. Python):
			
	\ttfamily
	\begin{grammar}
	
		<odd number> ::= <sign> <unsigned odd number>
		
		<sign> ::= + | - | <empty>
		
		<empty> ::= \
		
		<unsigned odd number> ::= <odd digit> | <natural number> <odd digit>
		
		<natural number> ::= <non-zero digit> | <natural number> <digit>
		
		<digit> ::= 0 | <non-zero digit>
		
		<non-zero digit> ::= 2 | 4 | 6 | 8 | <odd digit>
		
		<odd digit> ::= 1 | 3 | 5 | 7 | 9
		
	\end{grammar}
	
	
	
	
	\paragraph{}
	\rmfamily\
	
		e) Even number (..., -4, -2, 0, 2, 4, ..., 332, ..., 1022, ...). The answer:
		
	\ttfamily
	\begin{grammar}
	
		<even number> ::= <sign> <unsigned even number>
	
		<sign> ::= + | - | <empty>
	
		<empty> ::= \
	
		<unsigned even number> ::= <even digit> | <unsigned integer> <even digit>
	
		<unsigned integer> ::= <digit> | <unsigned integer> <digit>

		<digit> ::= 0 | 1 | 2 | 3 | 4 | 5 | 6 | 7 | 8 | 9
	
	\end{grammar}
	
		\rmfamily\
	
		Example of BNF definition of an even number in languages that do not support leading zeroes (e.g. Python):
		
	\ttfamily
	\begin{grammar}
	
		<even number> ::= <sign> <unsigned even number>
	
		<sign> ::= + | - | <empty>
	
		<empty> ::= \
	
		<unsigned even number> ::= <even digit> | <natural number> <even digit>
	
		<natural number> ::= <non-zero digit> | <natural number> <digit>
	
		<digit> ::= 0 | <non-zero digit>
	
		<non-zero digit> ::= 1 | 2 | 3 | 4 | 5 | 6 | 7 | 8 | 9
	
		<even digit> ::= 0 | 2 | 4 | 6 | 8
	
			
	\end{grammar}
	
	
	
	
	\paragraph{}
	\rmfamily
	
		f) Integer divisible by five (..., -10, 5, 0, 5, 10, ...). The answer:
		
	\ttfamily
	\begin{grammar}
	
		<integer div-by-5> ::= <sign> <unsigned int div-by-5>
		
		<sign> ::= + | - | <empty>
		
		<empty> ::= \
		
		<unsigned int div-by-5> ::= <div-by-5 suffix> | <unsigned integer> <div-by-5 suffix>
	
		<unsigned integer> ::= <digit> | <unsigned integer> <digit>
		
		<div-by-5 suffix> ::= 0 | 5

		<digit> ::= 0 | 1 | 2 | 3 | 4 | 5 | 6 | 7 | 8 | 9
		
	\end{grammar}
	
		\rmfamily
	
		Example of BNF definition of an integer divisible by 5 in languages that do not support leading zeroes (e.g. Python):
		
	\ttfamily
	\begin{grammar}
	
		<integer div-by-5> ::= <sign> <unsigned int div-by-5>
		
		<sign> ::= + | - | <empty>
		
		<empty> ::= \
		
		<unsigned int div-by-5> ::= <div-by-5 suffix> | <natural number> <div-by-5 suffix>
		
		<natural number> ::= <non-zero digit> | <natural number> <digit>
		
		<div-by-5 suffix> ::= 0 | 5
		
		<digit> ::= 0 | <non-zero digit>
		
		<non-zero digit> ::= 1 | 2 | 3 | 4 | 5 | 6 | 7 | 8 | 9
		
	\end{grammar}
	
	
	
	
	
	\paragraph{2. Show syntax diagrams for questions (a), ..., (f) of problem 1.}
	
		\includegraphics[width=\textwidth]{leadingzeroes4}
		\rmfamily
		Example of syntax diagrams for integers with no support of leading zeroes.
		\includegraphics[width=\textwidth-2pt]{noleadingzeroescrop6}
				
	\rmfamily
	
	
	
	
	\paragraph{3. Write a BNF definition of the syntax of (all possible) input statements in C++. }\
	
	\rmfamily\
	
		Following is an example of input statement in C++:
		\newline
	
		\texttt{cin	\lstinline[language=bash]/>>/	sclr	\lstinline[language=bash]/>>/	vec[2 * i - 1]	\lstinline[language=bash]/>>/	mat[f(i)][j + k]	\lstinline[language=bash]/>>/	t[i/3][j][k]; }
	
		The answer:
		
	\ttfamily
	
	
	
	\begin{grammar}
	
		<input statement> ::= cin <input arguments> ;
		
		<input arguments> ::= \lstinline[language=bash]/>>/ <input value> | <input arguments> \lstinline[language=bash]/>>/ <input value>
		
		<input value> ::= <struct member> | <array element> | <identifier>
		
		<struct member> ::= <identifier> . <identifier>
					   \alt <struct member> . <identifier> 
		
		<array element> ::= <identifier> <array indicies>
		
		<identifier> ::= <non-digit character> | <identifier> <digit>
		
		<array indicies> ::= <array index> | <array indicies> <array index>
		
		<array index> ::= [ <numerical expression> ]
		
		<numerical expression> ::= <arithmetic expression>
							  \alt <unary expression>
					    	  \alt <function call>
							  					
		<arithmetic expression> ::= <operand>
							   \alt <arithmetic expression> <operator> <arithmetic expression>

		<operator> ::= + | - | * | / | \% 

		<unary expression> ::= ++ <operand> | <operand> ++ 
							 \alt --- <operand> | <operand> ---
									 
		<operand> ::= <identifier> | <integer number> | <floating point number>
		
		<integer number> ::= <digits> | <digits> L | <digits> LL
		
		<floating point number> ::= <real number> | <real number> F
		
		<real number> ::= <digits> . <digits> | <digits> . | . <digits>
				
		<function call> ::= <identifier> ( <function arguments> )
		
		<function arguments> ::= <argument> | <function arguments> , <argument>
				
		<argument> ::= <function call> | <expression>
		
		<expression> ::= <numerical expression>
					\alt <string>
					\alt <character literal>
					\alt <void>
					
		<string> ::= "\"" <characters> "\""
		
		<character literal> ::= "\'" <char> "\'" | "\'" "\\" <char> "\'"

		<characters> ::= <char> | <characters> <char>
		
		<char> ::= <non-digit character> | <digit> | <whitespace>
		
		<whitespace> ::= ' '
		
		<non-digit character> ::= A | B | C | D | E | F | G | H | I | J
			  				 \alt K | L | M | N | O | P | Q | R | S | T 
			   			     \alt U | V | W | X | Y | Z | a | b | c | d 
			  				 \alt e | f | g | h | i | j | k | l | m | n
			   				 \alt o | p | q | r | s | t | u | v | w | x 
			  				 \alt y | z | _
		
		<digits> ::= <digit> | <digits> <digit>
		
		<digit> ::= 0 | 1 | 2 | 3 | 4 | 5 | 6 | 7 | 8 | 9
		
		<void> ::= <empty>
		
		<empty> ::= \

	\end{grammar}
	

	
	
	\rmfamily\
	
	\paragraph{4. Write a BNF definition of the syntax of (all possible) output statements in C++. }\
	
	\rmfamily\

	
		Following is an example of output statement in C++:
		\newline

	
		\texttt{ cout << 12.34 * a / rate << " " << 43.21 << " " }
		
		\texttt{ 	  << alpha + x[2*i-1] << " " << (p \&\& q) << " " }
		
		\texttt{      << pow(t[i][j],1.2) << " string " << 's' }
		
		\texttt{      << " " << myfun(x, sin(x+y), third_argument) ; } 
		
		\rmfamily\
		
		
		The answer (the definitions for auxiliary BNF productions are listed in the previous answer):
		
	\ttfamily\
	\begin{grammar}
	
		<output statement> ::= cout <output arguments> ;
		
		<output arguments> ::= \lstinline[language=bash]/<</ <output value> | <output arguments> \lstinline[language=bash]/<</ <output value>
		
		<output value> 	   ::= <expression>
		
	\end{grammar}
	

\end{document}