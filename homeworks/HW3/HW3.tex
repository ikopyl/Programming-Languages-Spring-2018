\documentclass{article}

\usepackage{listings}

\usepackage{minted}

\usepackage{syntax}

\usepackage[T1]{fontenc}

\usepackage{courier} % tt

\usepackage[english]{babel}
\usepackage[utf8]{inputenc}

\usepackage{pdfpages}

\usepackage{graphicx}
\graphicspath{ {images/} }

\usepackage{microtype}
\DisableLigatures[<]{encoding = T1}



\usepackage{listings}
\lstset{basicstyle=\ttfamily}



\begin{document}

	\setlength{\grammarparsep}{5pt plus 1pt minus 1pt} % increase separation between rules
%	\setlength{\grammarindent}{12em} % increase separation between LHS/RHS 
	\setlength{\grammarindent}{13em} % increase separation between LHS/RHS 
%	\setlength{\grammarindent}{5cm} 




	\begin{titlepage}
		\vspace*{\stretch{1.0}}
		\begin{center}
				\Large\textsc{CSc 600-01 (Section 1)}
				
				\Large\textbf{Homework 3 - Logic Programming in Prolog}\\

				\Large\textit{prepared by Ilya Kopyl}
				
		\end{center}	
		\vspace*{\stretch{2.0}}
	\end{titlepage}


	\title{\textsc{CSc 600 Homework 3 - Logic Programming in Prolog}}	
	\maketitle
	
		\textit{Homework is prepared by: Ilya Kopyl.}

		\textit{It is formatted in LaTeX, using TeXShop editor (under GNU GPL license).}
		
		\textit{Diagrams are created in LucidChart online editor (lucidchart.com).}

	\rmfamily\




	\paragraph{1. Write a PROLOG program that investigates family relationships using lists. The facts should be organized as follows:}\
	\rmfamily\\\
	
	\ttfamily
	\begin{minted}[fontsize=\small]{prolog} 
m([first_male_name, second_male_name, ..., last_male_name]).
f([first_female_name, second_female_name, ..., last_female_name]).
family( [father, mother, [child1, child2, ..., child_n]] ).
	\end{minted}
	
	\rmfamily
	Write rules that define the following relationships:
	
	\begin{lstlisting}[language=bash]
male(X)
female(X)
father, mother, parent
siblings1, siblings2
brother1, brother2
sister1, sister2
cousins
uncle, aunt
grandchild, grandson, granddaughter
greatgrandparent
ancestor
	\end{lstlisting}
	
	For each of these rules show an example of its use.
\paragraph{}\
	The answer is listed on the pages TBD through TBD.

\paragraph{}\
\paragraph{}\
\paragraph{}\
\paragraph{}\
\paragraph{}\
\paragraph{}\
\paragraph{}\



\noindent The code listing of maxlen function:
\ttfamily
	
\begin{minted}[fontsize=\small]{prolog} 
somePredicate(A, B) :-
    arbitraryPredicate(A, _, 1, 2),
    predicateWithAtom(someAtom),
    anotherPredicate(B, someAtom, myPredicate(A, _)),
    findall(X, ('testString'(X), myPredicate(A, X)), L1),
    member(A, L1),
    !.
\end{minted}
	

\paragraph{}\
\paragraph{}\
\paragraph{}\

		\rmfamily
		\noindent The code listing of main program:
		\begin{minted}[fontsize=\small]{prolog}
somePredicate(A, B) :-
    arbitraryPredicate(A, _, 1, 2),
    predicateWithAtom(someAtom),
    anotherPredicate(B, someAtom, myPredicate(A, _)),
    findall(X, ('testString'(X), myPredicate(A, X)), L1),
    member(A, L1),
    !.
		\end{minted}


\paragraph{}\
	
	\rmfamily\
	
		\noindent The result of the program execution:
		
	\ttfamily
	\begin{lstlisting}[language=bash]

Standard output:

	\end{lstlisting}
	
	
\paragraph{}\

	
	
	\rmfamily
	
	\paragraph{2. Integer plot function (find a smart way to code big integers) }\
	
	\rmfamily\
	
		Write a program BigInt(n) that displays an arbitrary positive integer n using big characters of size 7x7, as in the following example for BigInt(170):
				
	\ttfamily
	\begin{lstlisting}[language=bash]		
Standard output:
		 
	\end{lstlisting}
	
	\rmfamily\
	Write a demo main program that illustrates the work of BigInt(n) and prints the following sequence of big numbers 1, 12, 123, 1234,..., 1234567890, one below the other.
	\newline
	
	The answer is listed on the pages 7 through 9.
	
	
\paragraph{}\
\paragraph{}\
\paragraph{}\
\paragraph{}\
\paragraph{}\
\paragraph{}\
\paragraph{}\
\paragraph{}\
\paragraph{}\
\paragraph{}\
\paragraph{}\
\paragraph{}\

	
	\noindent The code listing of the two-dimensional array that stores bit pattern of each BigInt digit. It is declared in the global space (outside of any function).
	
	\begin{minted}[fontsize=\small]{prolog}
somePredicate(A, B) :-
    arbitraryPredicate(A, _, 1, 2),
    predicateWithAtom(someAtom),
    anotherPredicate(B, someAtom, myPredicate(A, _)),
    findall(X, ('testString'(X), myPredicate(A, X)), L1),
    member(A, L1),
    !.
	\end{minted}
	
	\rmfamily\
	\newline
	\noindent Main program, excluding the declaration of BIG_DIGITS array:
	
	\begin{minted}[fontsize=\small]{prolog}
somePredicate(A, B) :-
    arbitraryPredicate(A, _, 1, 2),
    predicateWithAtom(someAtom),
    anotherPredicate(B, someAtom, myPredicate(A, _)),
    findall(X, ('testString'(X), myPredicate(A, X)), L1),
    member(A, L1),
    !.
	\end{minted}
	

	
\paragraph{}\
\paragraph{}\
\paragraph{}\
\paragraph{}\
\paragraph{}\


	
	\rmfamily
	
	\paragraph{3. Array processing (elimination of three largest values) (one of many array reduction problems) }\
	
	\rmfamily\
	
		The array a(1..n) contains arbitrary integers. Write a function reduce(a, n) that reduces the array a(1..n) by eliminating from it all values that are equal to three largest different integers. For example, if a=(9, 1, 1, 6, 7, 1, 2, 3, 3, 5, 6, 6, 6, 6, 7, 9) then three largest different integers are 6, 7, 9, and after reduction the reduced array would be a=(1, 1, 1, 2, 3, 3, 5), n=7. The time complexity of the solution should be in O(n). 
		\newline
		
		The answer is listed on the pages 11 through 13.
		
	\ttfamily\
	
\paragraph{}\
\paragraph{}\
\paragraph{}\
\paragraph{}\
\paragraph{}\
\paragraph{}\
\paragraph{}\
\paragraph{}\
\paragraph{}\
\paragraph{}\
\paragraph{}\	
\paragraph{}\	
\paragraph{}\
\paragraph{}\	

	
		\rmfamily
		\noindent The code listing of the entire program for problem \#3:
		\begin{minted}[fontsize=\small]{prolog}
somePredicate(A, B) :-
    arbitraryPredicate(A, _, 1, 2),
    predicateWithAtom(someAtom),
    anotherPredicate(B, someAtom, myPredicate(A, _)),
    findall(X, ('testString'(X), myPredicate(A, X)), L1),
    member(A, L1),
    !.		\end{minted}
		
		
\paragraph{}\
\paragraph{}\

	\rmfamily

		
\end{document}