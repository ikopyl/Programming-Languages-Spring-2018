\documentclass{article}

\usepackage{listings}

\usepackage{minted}

\usepackage{syntax}

\usepackage[T1]{fontenc}

\usepackage{courier} % tt

\usepackage[english]{babel}
\usepackage[utf8]{inputenc}

\usepackage{pdfpages}

\usepackage{graphicx}
\graphicspath{ {images/} }

\usepackage{microtype}
\DisableLigatures[<]{encoding = T1}



\usepackage{listings}
\lstset{basicstyle=\ttfamily}



\begin{document}

	\setlength{\grammarparsep}{5pt plus 1pt minus 1pt} % increase separation between rules
%	\setlength{\grammarindent}{12em} % increase separation between LHS/RHS 
	\setlength{\grammarindent}{13em} % increase separation between LHS/RHS 
%	\setlength{\grammarindent}{5cm} 




	\begin{titlepage}
		\vspace*{\stretch{1.0}}
		\begin{center}
				\Large\textsc{CSc 600-01 (Section 1)}
				
				\Large\textbf{Homework 3 - Logic Programming in Prolog}\\

				\Large\textit{prepared by Ilya Kopyl}
				
		\end{center}	
		\vspace*{\stretch{2.0}}
	\end{titlepage}


	\title{\textsc{CSc 600 Homework 3 - Logic Programming in Prolog}}	
	\maketitle
	
		\textit{Homework is prepared by: Ilya Kopyl.}

		\textit{It is formatted in LaTeX, using TeXShop editor (under GNU GPL license).}
		
		\textit{Diagrams are created in LucidChart online editor (lucidchart.com).}

	\rmfamily\




	\paragraph{1. Write a PROLOG program that investigates family relationships using lists. The facts should be organized as follows:}\
	\rmfamily\\\
	
	\ttfamily
	\begin{minted}[fontsize=\small]{prolog} 
m([first_male_name, second_male_name, ..., last_male_name]).
f([first_female_name, second_female_name, ..., last_female_name]).
family( [father, mother, [child1, child2, ..., child_n]] ).
	\end{minted}
	
	\rmfamily
	Write rules that define the following relationships:
	
	\begin{lstlisting}[language=bash]
male(X)
female(X)
father, mother, parent
siblings1, siblings2
brother1, brother2
sister1, sister2
cousins
uncle, aunt
grandchild, grandson, granddaughter
greatgrandparent
ancestor
	\end{lstlisting}
	
	For each of these rules show an example of its use.
	\newline
	
	The answer is listed on the pages TBD through TBD.
	
	
\paragraph{}\	
	
	
	\rmfamily
		Partial diagram of the family tree of the British Royal family.
		\newline
		\includegraphics[width=\textwidth+120pt]{royalfamily2}
	


\paragraph{}\
\paragraph{}\
\paragraph{}\
\paragraph{}\
\paragraph{}\
\paragraph{}\



\noindent The code listing of maxlen function:
\ttfamily
	
\begin{minted}[fontsize=\small]{prolog} 
somePredicate(A, B) :-
    arbitraryPredicate(A, _, 1, 2),
    predicateWithAtom(someAtom),
    anotherPredicate(B, someAtom, myPredicate(A, _)),
    findall(X, ('testString'(X), myPredicate(A, X)), L1),
    member(A, L1),
    !.
\end{minted}
	

\paragraph{}\
\paragraph{}\
\paragraph{}\

		\rmfamily
		\noindent The code listing of main program:
		\begin{minted}[fontsize=\small]{prolog}
somePredicate(A, B) :-
    arbitraryPredicate(A, _, 1, 2),
    predicateWithAtom(someAtom),
    anotherPredicate(B, someAtom, myPredicate(A, _)),
    findall(X, ('testString'(X), myPredicate(A, X)), L1),
    member(A, L1),
    !.
		\end{minted}


\paragraph{}\
	
	\rmfamily\
	
		\noindent The result of the program execution:
		
	\ttfamily
	\begin{lstlisting}[language=bash]

Standard output:

	\end{lstlisting}
	
	
\paragraph{}\

	
	
	\rmfamily
	
	\paragraph{2. Write a PROLOG program that includes the following operations with lists: }\
	
	\ttfamily
	
	\begin{lstlisting}
membership testing (is an element member of a list?)
first element
last element
two adjacent elements
three adjacent elements
append list1 to list2 producing list3
delete element from a list
append element to a list
insert element in a list
compute the length of list
reverse a list
check whether a list is a palindrome
display a list
	\end{lstlisting}
	
	\rmfamily\
	For each of these operations write your implementation of the operation and show an example of its use. If a predicate already exists (predefined in Prolog), modify its name (e.g. myappend or append1).
	Lists to be processed can be created by an auxiliary program, defined as facts, or entered from the keyboard.
	\newline
	
	The answer is listed on the pages TBD through TBD.
	
	
\paragraph{}\
\paragraph{}\
\paragraph{}\
\paragraph{}\
\paragraph{}\
\paragraph{}\
\paragraph{}\
\paragraph{}\
\paragraph{}\
\paragraph{}\
\paragraph{}\
\paragraph{}\

	
	\noindent The code listing of the two-dimensional array that stores bit pattern of each BigInt digit. It is declared in the global space (outside of any function).
	
	\begin{minted}[fontsize=\small]{prolog}
somePredicate(A, B) :-
    arbitraryPredicate(A, _, 1, 2),
    predicateWithAtom(someAtom),
    anotherPredicate(B, someAtom, myPredicate(A, _)),
    findall(X, ('testString'(X), myPredicate(A, X)), L1),
    member(A, L1),
    !.
	\end{minted}
	
	\rmfamily\
	\newline
	\noindent Main program, excluding the declaration of BIG_DIGITS array:
	
	\begin{minted}[fontsize=\small]{prolog}
somePredicate(A, B) :-
    arbitraryPredicate(A, _, 1, 2),
    predicateWithAtom(someAtom),
    anotherPredicate(B, someAtom, myPredicate(A, _)),
    findall(X, ('testString'(X), myPredicate(A, X)), L1),
    member(A, L1),
    !.
	\end{minted}
	

	
\paragraph{}\
\paragraph{}\
\paragraph{}\
\paragraph{}\
\paragraph{}\


	
	\rmfamily
	
	\paragraph{3. Write a PROLOG program that solves the 8 queens problem (location of 8 queens on a chess board so that no queens have each other in check, i.e. are not located in the same row/column/diagonal). }\
	\newline
	\rmfamily\
	
	The answer is listed on the pages TBD through TBD.
	
	

		
\end{document}