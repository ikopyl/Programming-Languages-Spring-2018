\documentclass{article}

\usepackage{listings}

\usepackage{minted}

\usepackage{syntax}

\usepackage[T1]{fontenc}			% https://tex.stackexchange.com/questions/2369/why-do-the-less-than-symbol-and-the-greater-than-symbol-appear-wrong-as

\usepackage{courier} % tt

\usepackage[english]{babel}
\usepackage[utf8]{inputenc}

\usepackage{pdfpages}

\usepackage{graphicx}
\graphicspath{ {images/} }

\usepackage{microtype}
\DisableLigatures[<]{encoding = T1}



\usepackage{listings}
\lstset{basicstyle=\ttfamily}



\begin{document}

	\setlength{\grammarparsep}{5pt plus 1pt minus 1pt} % increase separation between rules
%	\setlength{\grammarindent}{12em} % increase separation between LHS/RHS 
	\setlength{\grammarindent}{13em} % increase separation between LHS/RHS 
%	\setlength{\grammarindent}{5cm} 




	\begin{titlepage}
		\vspace*{\stretch{1.0}}
		\begin{center}
				\Large\textsc{CSc 600-01 (Section 1)}
				
				\Large\textbf{Homework 2 - Procedural Programming}\\

				\Large\textit{prepared by Ilya Kopyl}
				
		\end{center}	
		\vspace*{\stretch{2.0}}
	\end{titlepage}


	\title{\textsc{CSc 600 Homework 2 - Procedural Programming}}	
	\maketitle
	
		\textit{Homework is prepared by: Ilya Kopyl.}

		\textit{It is formatted in LaTeX, using TeXShop editor (under GNU GPL license).}
		
		% \textit{Syntax diagrams are created in LucidChart online editor (lucidchart.com).}


	\rmfamily\




	\paragraph{1. Plateau program (max sequence length) (a combinatorial algorithm)}\
	\rmfamily\\\
	
		The array a(1..n) contains sorted integers. Write a function maxlen(a,n) that returns the length of the longest sequence of identical numbers (for example, if a={ 1, 1, 1, 2, 3, 3, 5, 6, 6, 6, 6, 7, 9 } then maxlen returns 4 because the longest sequence 6, 6, 6, 6 contains 4 numbers. Write a demo main program for testing the work of maxlen. Explain your solution, and insert comments in your program. The time complexity of the solution should be in O(n).
		\newline
		
		The answer is listed on the pages 2 through TBD.

\paragraph{}\
\paragraph{}\
\paragraph{}\
\paragraph{}\
\paragraph{}\
\paragraph{}\
\paragraph{}\
\paragraph{}\


\noindent The code listing of maxlen function:
\ttfamily
	
\begin{minted}[fontsize=\small]{c} 


unsigned int maxlen(int *a, unsigned int n)
{
    // handling the edge cases - arrays of size 0 and 1:
    if (n < 2)                                      
        return n;

    unsigned int max_count, current_count, i;
    i = max_count = 0;
    current_count = 1;

    printf("    a[%d]=%d; \tcurrent_count=%d; \tmax_count=%d\n",
           i, a[i], current_count, max_count);

    for (i = 1; i < n; ++i)
    {
        if (a[i] == a[i-1])         // counting the current sequence
        {
            current_count++;

            // checking whether the longest sequence is at the end of array
            if(i == n-1 && current_count > max_count)
                max_count = current_count;
        }
        else            	    // starting the count of the new sequence
        {
            // before resetting the counter, save it's value if it is above threshold
            if (current_count > max_count)
                max_count = current_count;

            // exit the loop if max_count is sufficiently large
            if (max_count >= n-i)
                break;
                
            current_count = 1;
        }

        printf("    a[%d]=%d; \tcurrent_count=%d; \tmax_count=%d\n",
               i, a[i], current_count, max_count);
    }
    return max_count;
}
\end{minted}	
	

\paragraph{}\
\paragraph{}\
\paragraph{}\

		\noindent The code listing of main program:
		\begin{minted}[fontsize=\small]{c}
#include <stdio.h>
#include <assert.h>
#include "functions.h"

unsigned int maxlen(int *, unsigned int);

int main()
{
    int result = 0;

    int a[13] = { 1, 1, 1, 2, 3, 3, 5, 6, 6, 6, 6, 7, 9 };
    printf("Array a:    ");
    printIntArray(a, sizeof(a));

    result = maxlen(a, 13);
    printf("Max sequence length of array a = %d\n\n", result);
    assert(result == 4);


    int b[0] = {};            // test case: an empty array
    printf("Array b:    ");
    printIntArray(b, sizeof(b));

    result = maxlen(b, 0);
    printf("Max sequence length of array b = %d\n\n", result);
    assert(result == 0);


    int c[1] = { 12 };        // test case: 1 element in the array
    printf("Array c:    ");
    printIntArray(c, sizeof(c));

    result = maxlen(c, 1);
    printf("Max sequence length of array c = %d\n\n", result);
    assert(result == 1);


    // testing the early loop exit:
    int d[6] = { 16, 16, 16, 18, 18, 20 };
    printf("Array d:    ");
    printIntArray(d, sizeof(d));

    result = maxlen(d, 6);
    printf("Max sequence length of array d = %d\n\n", result);
    assert(result == 3);


    int e[2] = { 0, 0 };        // test case: 2 elements with the same value
    printf("Array e:    ");
    printIntArray(e, sizeof(e));

    result = maxlen(e, 2);
    printf("Max sequence length of array e = %d\n\n", result);
    assert(result == 2);


    int f[2] = { 0, 1 };        // test case: 2 elements with different values
    printf("Array f: ");
    printIntArray(f, sizeof(f));

    result = maxlen(f, 2);
    printf("Max sequence length of array f = %d\n\n", result);
    assert(result == 1);


    int g[4] = { 1, 2, 3, 3 };  // test case: the longest sequence ends with array
    printf("Array g:    ");
    printIntArray(g, sizeof(g));

    result = maxlen(g, 4);
    printf("Max sequence length of array g = %d\n\n", result);

    return 0;
}
		\end{minted}



\paragraph{}\
\paragraph{}\
\paragraph{}\
\paragraph{}\
\paragraph{}\
\paragraph{}\
\paragraph{}\
\paragraph{}\
\paragraph{}\



	
	\rmfamily\
	
		\noindent The result of the program execution:
		
	\ttfamily
	\begin{lstlisting}[language=bash]

Array a:    1  1  1  2  3  3  5  6  6  6  6  7  9
    a[0]=1; 	current_count=1; 	max_count=0
    a[1]=1; 	current_count=2; 	max_count=0
    a[2]=1; 	current_count=3; 	max_count=0
    a[3]=2; 	current_count=1; 	max_count=3
    a[4]=3; 	current_count=1; 	max_count=3
    a[5]=3; 	current_count=2; 	max_count=3
    a[6]=5; 	current_count=1; 	max_count=3
    a[7]=6; 	current_count=1; 	max_count=3
    a[8]=6; 	current_count=2; 	max_count=3
    a[9]=6; 	current_count=3; 	max_count=3
    a[10]=6; 	current_count=4; 	max_count=3
Max sequence length of array a = 4

Array b:
Max sequence length of array b = 0

Array c:    12
Max sequence length of array c = 1

Array d:    16  16  16  18  18  20
    a[0]=16; 	current_count=1; 	max_count=0
    a[1]=16; 	current_count=2; 	max_count=0
    a[2]=16; 	current_count=3; 	max_count=0
Max sequence length of array d = 3

Array e:    0  0
    a[0]=0; 	current_count=1; 	max_count=0
    a[1]=0; 	current_count=2; 	max_count=2
Max sequence length of array e = 2

Array f: 0  1
    a[0]=0; 	current_count=1; 	max_count=0
Max sequence length of array f = 1

Array g:    1  2  3  3
    a[0]=1; 	current_count=1; 	max_count=0
    a[1]=2; 	current_count=1; 	max_count=1
    a[2]=3; 	current_count=1; 	max_count=1
    a[3]=3; 	current_count=2; 	max_count=2
Max sequence length of array g = 2

	\end{lstlisting}
	
	


	
	
	\rmfamily
	
	\paragraph{2. Integer plot function (find a smart way to code big integers) }\
	
	\rmfamily\
	
		Write a program BigInt(n) that displays an arbitrary positive integer n using big characters of size 7x7, as in the following example for BigInt(170):
				
	\ttfamily
	\begin{lstlisting}[language=bash]		
	   @@ 	 @@@@@@@  @@@@@  
	  @@@  	      @@ @@   @@ 
	   @@  	     @@  @@   @@ 
	   @@  	    @@   @@   @@ 
	   @@  	   @@    @@   @@ 
	   @@  	  @@     @@   @@ 
	 @@@@@@	 @@       @@@@@  
		 
	\end{lstlisting}
	
	\rmfamily\
	Write a demo main program that illustrates the work of BigInt(n) and prints the following sequence of big numbers 1, 12, 123, 1234,..., 1234567890, one below the other.
	\newline
	
	The answer is listed on the pages TBD through TBD.
	
	
\paragraph{}\
\paragraph{}\
\paragraph{}\
\paragraph{}\
\paragraph{}\
\paragraph{}\
\paragraph{}\
\paragraph{}\
\paragraph{}\
\paragraph{}\
\paragraph{}\
\paragraph{}\

	
	\noindent The code listing of the two-dimensional array that stores bit pattern of each BigInt digit. It is declared in the global space (outside of any function).
	
	\begin{minted}[fontsize=\small]{c}
#define NUMBER_OF_ROWS 8

/**
 * Digits are stored as bit patterns of 8-bit unsigned integer (char) numbers.
 *
 * Each digit requires just 8 bytes of storage - which is polynomially smaller
 * than the storage in brute-force approach where each digit is represented by
 * a 2D array of 8x8 characters, with 64 bytes of storage per digit.
 */
const unsigned char BIG_DIGITS[NUMBER_OF_ROWS][10] = 
{
    {   // row 0 of all 10 digits
        0b00000000u, 0b00000000u, 0b00000000u, 0b00000000u, 0b00000000u,
        0b00000000u, 0b00000000u, 0b00000000u, 0b00000000u, 0b00000000u
    },
    {   // row 1 of all 10 digits
        0b00111110u, 0b00001100u, 0b00011110u, 0b00011110u, 0b00000110u,
        0b00111111u, 0b00011110u, 0b01111111u, 0b00011110u, 0b00011110u
    },
    {   // row 2 of all 10 digits
        0b01100011u, 0b00011100u, 0b00110011u, 0b00110011u, 0b00001110u,
        0b00110000u, 0b00110011u, 0b00000011u, 0b00110011u, 0b00110011u
    },
    {   // row 3 of all 10 digits
        0b01100011u, 0b00001100u, 0b00110011u, 0b00000011u, 0b00010110u,
        0b00110000u, 0b00110000u, 0b00000110u, 0b00110011u, 0b00110011u
    },
    {   // row 4 of all 10 digits
        0b01100011u, 0b00001100u, 0b00000110u, 0b00001100u, 0b00110110u,
        0b00111110u, 0b00111110u, 0b00001100u, 0b00011110u, 0b00011111u
    },
    {   // row 5 of all 10 digits
        0b01100011u, 0b00001100u, 0b00001100u, 0b00000011u, 0b01100110u,
        0b00000011u, 0b00110011u, 0b00011000u, 0b00110011u, 0b00000011u
    },
    {   // row 6 of all 10 digits
        0b01100011u, 0b00001100u, 0b00011000u, 0b00110011u, 0b01111111u,
        0b00000011u, 0b00110011u, 0b00110000u, 0b00110011u, 0b00110011u
    },
    {   // row 7 of all 10 digits
        0b00111110u, 0b00111111u, 0b00111111u, 0b00011110u, 0b00000110u,
        0b00111110u, 0b00011110u, 0b01100000u, 0b00011110u, 0b00011110u
    }
};
	\end{minted}
	
	\rmfamily\
	\newline
	\noindent Main program, excluding the declaration of BIG_DIGITS array:
	
	\begin{minted}[fontsize=\small]{c}
#include <stdio.h>
#include <math.h>

void BigInt(unsigned int);
unsigned int getNumberOfDigits(unsigned int);

#define NUMBER_OF_BITS 8
#define NUMBER_OF_ROWS 8

/** BIG_DIGITS[][] is declared here; its declaration is listed on the previous page */

int main() 
{
    BigInt(1);
    BigInt(12);
    BigInt(123);
    BigInt(1234);
    BigInt(1234567890);
    return 0;
}
void BigInt(unsigned int n)
{
    unsigned int numOfDigits, c;
    c = numOfDigits = getNumberOfDigits(n);
    int decimals[numOfDigits];

    // decomposing the number into an array of decimal digits
    do {
        decimals[c-1] = n % 10;
        c--;
    } while ((n /= 10));

    // printing all digits at once, row by row
    for (int row = 0; row < NUMBER_OF_ROWS; row++)
    {
        for (int digit = 0; digit < numOfDigits; digit++)
            /* iteratively extracting bit pattern of each char of BIG_DIGITS array
               and printing it, starting from the most significant bit first */
            for (int bit = NUMBER_OF_BITS-1; bit >= 0; bit--)
                printf("%c", 
                ((BIG_DIGITS[row][decimals[digit]] >> bit) & 1) == 1 ? '@' : ' ');
        puts("");    // adds a newline character at the end of each printed row
    }
}
unsigned int getNumberOfDigits(unsigned int n) 
{
    return (unsigned int) log10(n) + 1;        // floor of log10(n) + 1
}
	\end{minted}
	
	
	\noindent The result of the program execution:
	
%	\begin{lstlisting}[basicstyle=\tiny, language=bash]%[basicstyle=\tiny, %or \small or \footnotesize etc.]
	\begin{lstlisting}[basicstyle=\fontsize{7}{9}\ttfamily, language=bash]%[basicstyle=\tiny, %or \small or \footnotesize etc.]


    @@
   @@@
    @@
    @@
    @@
    @@
  @@@@@@

    @@     @@@@
   @@@    @@  @@
    @@    @@  @@
    @@       @@
    @@      @@
    @@     @@
  @@@@@@  @@@@@@

    @@     @@@@    @@@@
   @@@    @@  @@  @@  @@
    @@    @@  @@      @@
    @@       @@     @@
    @@      @@        @@
    @@     @@     @@  @@
  @@@@@@  @@@@@@   @@@@

    @@     @@@@    @@@@      @@
   @@@    @@  @@  @@  @@    @@@
    @@    @@  @@      @@   @ @@
    @@       @@     @@    @@ @@
    @@      @@        @@ @@  @@
    @@     @@     @@  @@ @@@@@@@
  @@@@@@  @@@@@@   @@@@      @@

    @@     @@@@    @@@@      @@   @@@@@@   @@@@  @@@@@@@   @@@@    @@@@   @@@@@
   @@@    @@  @@  @@  @@    @@@   @@      @@  @@      @@  @@  @@  @@  @@ @@   @@
    @@    @@  @@      @@   @ @@   @@      @@         @@   @@  @@  @@  @@ @@   @@
    @@       @@     @@    @@ @@   @@@@@   @@@@@     @@     @@@@    @@@@@ @@   @@
    @@      @@        @@ @@  @@       @@  @@  @@   @@     @@  @@      @@ @@   @@
    @@     @@     @@  @@ @@@@@@@      @@  @@  @@  @@      @@  @@  @@  @@ @@   @@
  @@@@@@  @@@@@@   @@@@      @@   @@@@@    @@@@  @@        @@@@    @@@@   @@@@@	 
	\end{lstlisting}
	
	
\paragraph{}\
\paragraph{}\
\paragraph{}\
\paragraph{}\
\paragraph{}\


	
	\rmfamily
	
	\paragraph{3. Array processing (elimination of three largest values) (one of many array reduction problems) }\
	
	\rmfamily\
	
		The array a(1..n) contains arbitrary integers. Write a function reduce(a, n) that reduces the array a(1..n) by eliminating from it all values that are equal to three largest different integers. For example, if a=(9, 1, 1, 6, 7, 1, 2, 3, 3, 5, 6, 6, 6, 6, 7, 9) then three largest different integers are 6, 7, 9, and after reduction the reduced array would be a=(1, 1, 1, 2, 3, 3, 5), n=7. The time complexity of the solution should be in O(n). 
		\newline
		
		The answer is listed on the pages TBD through TBD.
		
	\ttfamily\
	
\paragraph{}\
\paragraph{}\
\paragraph{}\
\paragraph{}\
\paragraph{}\
\paragraph{}\
\paragraph{}\
\paragraph{}\
\paragraph{}\
\paragraph{}\
\paragraph{}\	
\paragraph{}\	
\paragraph{}\
\paragraph{}\	

	
		\noindent The code listing of the entire program for problem \#3:
		\begin{minted}[fontsize=\small]{c}
#include <stdio.h>
#include "functions.h"

unsigned int reduce(int *, unsigned int);
void findTop3MaxValuesInArray(int *, unsigned int, int *, int *, int *);
void nullifyTop3MaxValuesInArray(int *, unsigned int, int, int, int);
unsigned int moveZeroesToEndOfArray(int *, unsigned int);

int main()
{
    int a[16] = { 9, 1, 1, 6, 7, 1, 2, 3, 3, 5, 6, 6, 6, 6, 7, 9 };

    puts("Original array: ");
    printIntArray(a, sizeof(a));

    int reducedN = reduce(a, 16);

    puts("Reduced array with original bounds: ");
    printIntArray(a, sizeof(a));

    printf("Reduced array with new bounds (n = %d): \n", reducedN);
    printIntArray(a, reducedN * sizeof(int));

    return 0;
}

unsigned int reduce(int * a, unsigned int n)
{
    int max1, max2, max3;
    max1 = max2 = max3 = 0;

    findTop3MaxValuesInArray(a, 16, &max1, &max2, &max3);
    nullifyTop3MaxValuesInArray(a, 16, max1, max2, max3);

    return moveZeroesToEndOfArray(a, n);
}

/** Setting all occurrences of numbers max1, max2, max3 in the array to zero. */
void nullifyTop3MaxValuesInArray(int * a, unsigned int n, int max1, int max2, int max3)
{
    unsigned int i;
    for (i = 0; i < n; i++)
        if (a[i] == max1 || a[i] == max2 || a[i] == max3)
            a[i] = 0;
}


/** Finding the first 3 maximum values of an array. */
void findTop3MaxValuesInArray(int * a, unsigned int n, int *max1, int *max2, int *max3)
{
    *max1 = *max2 = *max3 = 0;
    unsigned int i;
    for (i = 0; i < n; i++)
    {
        if (a[i] > *max1)
        {
            *max3 = *max2;
            *max2 = *max1;
            *max1 = a[i];
        }
        else if (a[i] > *max2 && a[i] < *max1)
        {
            *max3 = *max2;
            *max2 = a[i];
        }
        else if (a[i] > *max3 && a[i] < *max2)
            *max3 = a[i];
    }
}

/** Moves all zeroes to the end of array by copying all non-zero
 * values to the beginning of the array and returning the total
 * amount of non-zero values (i.e. the size of reduced array). */
unsigned int moveZeroesToEndOfArray(int * a, unsigned int n)
{
    unsigned int indexOfNullValue, sizeOfReducedArray;
    indexOfNullValue = 0;

    // copying all non-zero values to the beginning of array
    unsigned int i;
    for (i = 0; i < n; i++)
        if (a[i] != 0)
            a[indexOfNullValue++] = a[i];

    sizeOfReducedArray = indexOfNullValue;

    // nullifying the remaining part of the array
    while (indexOfNullValue < n)
        a[indexOfNullValue++] = 0;

    // returning the size of reduced array
    return sizeOfReducedArray;
}
		\end{minted}
		
\paragraph{}\
\paragraph{}\
		
		\rmfamily
		\noindent The result of the program execution:
		
		\ttfamily
		\begin{lstlisting}[language=bash]
$ gcc -Wall -std=c99 hw2-problem3.c functions.c -O3
$ ./a.out
Original array:
9  1  1  6  7  1  2  3  3  5  6  6  6  6  7  9
Reduced array with original bounds:
1  1  1  2  3  3  5  0  0  0  0  0  0  0  0  0
Reduced array with new bounds (n = 7):
1  1  1  2  3  3  5
		\end{lstlisting}
		
\paragraph{}\	
\paragraph{}\
\paragraph{}\

	\rmfamily
	
	\paragraph{4. Iteration versus recursion (an opportunity for performance measurement) }\
	
	\rmfamily\
	
		Make a sorted integer array a[i]=i, i=0,...,n-1.  Let bs(a, n, x) be a binary search program that returns the index i of the array a[0..n-1] where a[i]=x. Obviously, the result is bs(a, n, x)=x, and the binary search function can be tested using the loop
		
		\begin{minted}[fontsize=\small]{c}
for (j=0; j < K; j++)
    for (i=0; i < n; i++)
        if (bs(a, n, i) != i)
            cout << "\nERROR";
		\end{minted}
		
		
		Select the largest n your software can support and then K so that this loop with an iterative version of bs runs 3 seconds or more. Then measure and compare this run time and the run time of the loop that uses a recursive version of bs. Compare these run times using maximum compiler optimization (release version) and the slowest version (minimum optimization or the debug version). If you use a laptop, make measurements using AC power, and then the same measurements using only the battery. What conclusions can you derive from these experiments? Who is faster? Why?
		\newline
		
		The answer is listed on the pages TBD through TBD.
	

\paragraph{}\
\paragraph{}\

		\rmfamily\
		
		\noindent The code listing of the entire program for problem \#4:
		\begin{minted}[fontsize=\small]{c}
#include <stdio.h>
#include <time.h>

#define K 1000                      // system-dependent constant

void initializeArray(int *, int);
int ibs(int *, int, int);
int rbs(int *, int, int, int);
double ibsTest(int *, int);
double rbsTest(int *, int);

int main() 
{
    int sizeOfArray = 65535;        // 2^16-1, staying conservative
    int a[sizeOfArray];

    initializeArray(a, sizeOfArray);

    printf("Running time of iterative Binary Search: %f seconds.\n", 
           ibsTest(a, sizeOfArray));
    printf("Running time of recursive Binary Search: %f seconds.\n",
           rbsTest(a, sizeOfArray));
           
    puts("Benchmarking is complete!");

    return 0;
}

void initializeArray(int *a, int n)
{
    for (int i = 0; i < n; i++)
        a[i] = i;
}














/**
 * Iterative implementation of Binary Search
 */
int ibs(int *a, int n, int value)
{
    int itemLocation = -1;
    int low, mid, high;
    low = 0;
    high = n;

    while (high >= low && itemLocation == -1)
    {
        mid = (low + high) / 2;
        if (value == a[mid])
            itemLocation = mid;
        else if (value < a[mid])
            high = mid - 1;
        else
            low = mid + 1;
    }
    return itemLocation;
}

/**
 * Recursive implementation of Binary Search
 */
int rbs(int *a, int low, int high, int value)
{
    if (high < low)
        return -1;		// the value not found
    else
    {
        int mid = low + (high - low) / 2;
        if (a[mid] == value)
            return mid;
        else if (a[mid] > value)
            return rbs(a, low, mid-1, value);
        else
            return rbs(a, mid+1, high, value);
    }
}









/**
 * Function for testing the performance
 * of iterative binary search.
 */
double ibsTest(int *a, int n)
{
    clock_t start_t, end_t, running_time = 0;
    int i, j;

    start_t = clock();
    for (j = 0; j < K; j++)
        for (i = 0; i < n; i++)
        {
            if (ibs(a, n, i) != i)
                puts("ERROR");
        }
    end_t = clock();
    running_time += (end_t - start_t);

    return (double) running_time/CLOCKS_PER_SEC;
}

/**
 * Function for testing the performance
 * of recursive binary search.
 */
double rbsTest(int *a, int n)
{
    clock_t start_t, end_t, running_time = 0;
    int i, j;

    start_t = clock();
    for (j = 0; j < K; j++)
        for (i = 0; i < n; i++)
        {
            if (rbs(a, 0, n, i) != i)
                puts("ERROR");
        }
    end_t = clock();
    running_time += (end_t - start_t);

    return (double) running_time/CLOCKS_PER_SEC;
}

		\end{minted}


\paragraph{}\
\paragraph{}\
\paragraph{}\




\noindent The result of the program execution with different setup \& compiler optimizations:



	
	\ttfamily
	\begin{lstlisting}[basicstyle=\small, language=bash][language=bash]

# connected to charger, no Wi-Fi, no monitors connected:

$ gcc -Wall -std=c99 hw2-problem4.c -O0
$ ./a.out
Running time of iterative Binary Search: 12.815268 seconds.
Running time of recursive Binary Search: 13.416952 seconds.
Benchmarking is complete!

$ gcc -Wall -std=c99 hw2-problem4.c -O1
$ ./a.out
Running time of iterative Binary Search: 3.122441 seconds.
Running time of recursive Binary Search: 3.351381 seconds.
Benchmarking is complete!

$ gcc -Wall -std=c99 hw2-problem4.c -O2
$ ./a.out
Running time of iterative Binary Search: 2.831430 seconds.
Running time of recursive Binary Search: 2.775777 seconds.
Benchmarking is complete!


$ gcc -Wall -std=c99 hw2-problem4.c -O3
$ ./a.out
Running time of iterative Binary Search: 3.120335 seconds.
Running time of recursive Binary Search: 2.861631 seconds.
Benchmarking is complete!
	\end{lstlisting}

\paragraph{}\
	\rmfamily
	Initially, with minimal compiler optimization, the implementation of recursive Binary Search runs comparatively slower than iterative Binary Search.
	However, as we increase the intensity of the compiler optimization, we shall see that the recursive Binary Search outperforms its iterative counterpart. One counterintuitive observation is related to the program performance while a laptop is powered by only a battery: as you can see, it runs a bit faster than while plugged in.

	\ttfamily
	\begin{lstlisting}[basicstyle=\small, language=bash][language=bash]	
	
$ gcc -Wall -std=c99 hw2-problem4.c -O0
$ ./a.out
Running time of iterative Binary Search: 12.409296 seconds.
Running time of recursive Binary Search: 13.290153 seconds.
Benchmarking is complete!

$ gcc -Wall -std=c99 hw2-problem4.c -O3
$ ./a.out
Running time of iterative Binary Search: 3.029139 seconds.
Running time of recursive Binary Search: 2.747411 seconds.
Benchmarking is complete!
	\end{lstlisting}

\paragraph{}\

	
	\rmfamily
	
	\paragraph{5. Iteration versus recursion (another opportunity for performance measurement) }\
	
	\rmfamily\
	
		Write a recursive function Frec(n) that computes Fibonacci numbers. Then write an iterative version of Fibonacci number function Fit(n). Functions Frec(n) and Fit(n) return the same value but with different performance.
		\newline
		
		Write the main program that discovers the value N10 so that Frec(N10) runs on your machine exactly 10 seconds. Then measure the run time of Fit(N10) and compute how many times is Fit(N10) faster than Frec(N10). Show what is N10 on your machine.
		\newline
		
		\noindent Notes:
		
		1. When you measure the speed, your machine should be disconnected from the Internet, it should use the AC power supply, and it should run only one program (your performance measurement program).
		
		2. In C++ you can measure current time in seconds using the following function:
		
		\begin{minted}[fontsize=\small]{c}
double sec(void)
{
    return double(clock()) / double(CLOCKS_PER_SEC);
}
		\end{minted}
		
		To measure the run time of fast programs you must repeat them many times inside a loop. Take care to eliminate the overhead generated by the loop.
		\newline
		
		The answer is listed on the pages TBD through TBD.

\paragraph{}\
\paragraph{}\
\paragraph{}\
\paragraph{}\
\paragraph{}\
\paragraph{}\
\paragraph{}\
\paragraph{}\


		\noindent The code listing of the entire program for problem \#5: 
		\begin{minted}[fontsize=\small]{c}
#include <stdio.h>
#include <zconf.h>
#include <time.h>

int Frec(int);
int Fit(int);

double findN(int *);

// using the function pointer to avoid unnecessary code repetition:
double benchmarkFibFunction(int (*f)(int), int);


int main()
{
    puts("Depending on the compiler optimizations, results may vary. Please wait...");

    int threshold = 10;
    // !!! the value of threshold will become an index of Nth Fibonachi term
    double runningTimeFrec = findN(&threshold);
    printf("N10 = %d\n", threshold);
    printf("Running time of Frec(%d) is %f seconds.\n", 
           threshold, runningTimeFrec);

    double runningTimeFit = benchmarkFibFunction(Fit, threshold);
    printf("Running time of Fit(%d) is %f seconds.\n", 
           threshold, runningTimeFit);

    double speedupFactor = runningTimeFrec / runningTimeFit;

    printf("Fit(%1$d) is %2$.2f times faster than Frec(%1$d).\n",
           threshold, speedupFactor);

    return 0;
}












/** Function iteratively runs benchmarks against recursive Fibonacci function
 * until the running time exceeds the given time threshold in seconds.
 * @return index of nth Fibonacci term is returned through parameter (dirty hack).
 * @return the time it took to compute nth Fibonacci term. */
double findN(int *timeThreshold) {
    int n = 0;
    double runningTime = 0;
    while ((runningTime = benchmarkFibFunction(Frec, ++n)) < *timeThreshold);
    *timeThreshold = n;         
    return runningTime;
}

/** Function measures the performance of both Fibonacci functions.
 *
 * @param f accepts any function that takes and returns an int value.
 * @param n int value that will be passed as a parameter to f
 * @return returns running time in seconds of function f for input n. */
double benchmarkFibFunction(int (*f)(int), int n)
{
    clock_t start_t, end_t = 0;
    start_t = clock();
    (*f)(n);
    end_t = clock();
    return (double) (end_t - start_t) / CLOCKS_PER_SEC;
}

/** Recursive implementation of function
 * computing Fibonacci numbers. */
int Frec(int n)
{
    return n <= 1 ? n : Frec(n - 1) + Frec(n - 2);
}
/** Iterative implementation of function computing Fibonacci numbers. */
int Fit(int n)
{
    int first, second, temp, i;
    first = 0;
    second = 1;

    if (n <= 1)
        return n;
    else
        for (i = 2; i <= n; i++)
        {
            temp = first + second;
            first = second;
            second = temp;
        }
    return temp;        // the n-th value in Fibonacci sequence
}
		\end{minted}
		
		
\paragraph{}\
\paragraph{}\
\paragraph{}\

	\rmfamily
	\noindent The result of the program execution with different compiler optimizations:
	
	\ttfamily
	\begin{lstlisting}[language=bash]
	
$ gcc -std=c99 -Wall hw2-problem5.c -O0
$ ./a.out && ./a.out && ./a.out
Depending on the compiler optimizations, the results may vary. Please wait...
N10 = 46
Running time of Frec(46) is 12.659873 seconds.
Running time of Fit(46) is 0.000001 seconds.
Fit(46) is 12659873.00 times faster than Frec(46).
Depending on the compiler optimizations, the results may vary. Please wait...
N10 = 46
Running time of Frec(46) is 12.380931 seconds.
Running time of Fit(46) is 0.000002 seconds.
Fit(46) is 6190465.50 times faster than Frec(46).
Depending on the compiler optimizations, the results may vary. Please wait...
N10 = 46
Running time of Frec(46) is 12.498848 seconds.
Running time of Fit(46) is 0.000001 seconds.
Fit(46) is 12498848.00 times faster than Frec(46).

$ gcc -std=c99 -Wall hw2-problem5.c -O1
$ ./a.out && ./a.out && ./a.out
Depending on the compiler optimizations, the results may vary. Please wait...
N10 = 47
Running time of Frec(47) is 15.694785 seconds.
Running time of Fit(47) is 0.000001 seconds.
Fit(47) is 15694785.00 times faster than Frec(47).
Depending on the compiler optimizations, the results may vary. Please wait...
N10 = 47
Running time of Frec(47) is 15.843899 seconds.
Running time of Fit(47) is 0.000002 seconds.
Fit(47) is 7921949.50 times faster than Frec(47).
Depending on the compiler optimizations, the results may vary. Please wait...
N10 = 47
Running time of Frec(47) is 15.601271 seconds.
Running time of Fit(47) is 0.000001 seconds.
Fit(47) is 15601271.00 times faster than Frec(47).
	\end{lstlisting}
	
	
		
\end{document}