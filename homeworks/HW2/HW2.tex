\documentclass{article}

\usepackage{listings}

\usepackage{minted}

\usepackage{syntax}

\usepackage[T1]{fontenc}			% https://tex.stackexchange.com/questions/2369/why-do-the-less-than-symbol-and-the-greater-than-symbol-appear-wrong-as

\usepackage{courier} % tt

\usepackage[english]{babel}
\usepackage[utf8]{inputenc}

\usepackage{pdfpages}

\usepackage{graphicx}
\graphicspath{ {images/} }

\usepackage{microtype}
\DisableLigatures[<]{encoding = T1}



\usepackage{listings}
\lstset{basicstyle=\ttfamily}



\begin{document}

	\setlength{\grammarparsep}{5pt plus 1pt minus 1pt} % increase separation between rules
%	\setlength{\grammarindent}{12em} % increase separation between LHS/RHS 
	\setlength{\grammarindent}{13em} % increase separation between LHS/RHS 
%	\setlength{\grammarindent}{5cm} 




	\begin{titlepage}
		\vspace*{\stretch{1.0}}
		\begin{center}
				\Large\textsc{CSc 600-01 (Section 1)}
				
				\Large\textbf{Homework 2 - Procedural Programming}\\

				\Large\textit{prepared by Ilya Kopyl}
				
		\end{center}	
		\vspace*{\stretch{2.0}}
	\end{titlepage}


	\title{\textsc{CSc 600 Homework 2 - Procedural Programming}}	
	\maketitle
	
		\textit{Homework is prepared by: Ilya Kopyl.}

		\textit{It is formatted in LaTeX, using TeXShop editor (under GNU GPL license).}
		
		% \textit{Syntax diagrams are created in LucidChart online editor (lucidchart.com).}


	\rmfamily\




	\paragraph{1. Plateau program (max sequence length) (a combinatorial algorithm)}\
	\rmfamily\\\
	
		The array a(1..n) contains sorted integers. Write a function maxlen(a,n) that returns the length of the longest sequence of identical numbers (for example, if a={ 1, 1, 1, 2, 3, 3, 5, 6, 6, 6, 6, 7, 9 } then maxlen returns 4 because the longest sequence 6, 6, 6, 6 contains 4 numbers. Write a demo main program for testing the work of maxlen. Explain your solution, and insert comments in your program. The time complexity of the solution should belong to O(n).

\paragraph{}\
\paragraph{}\
\paragraph{}\
\paragraph{}\
\paragraph{}\
\paragraph{}\
\paragraph{}\
\paragraph{}\


\noindent A code listing of implementation of maxlen function:
	\ttfamily
	
\begin{minted}[fontsize=\small]{c} 


unsigned int maxlen(int *a, unsigned int n)
{
    // handling the edge cases of arrays of size 0 and 1
    if (n < 2)                                      
        return n;

    unsigned int max_count, current_count, i;
    i = max_count = 0;
    current_count = 1;

    printf("\ta[%d]=%d; \tcurrent_count=%d; \tmax_count=%d\n",
           i, a[i], current_count, max_count);

    for (i = 1; i < n; ++i)
    {
        if (a[i] != a[i-1])    // starting the count of the new sequence:
        {
            if (current_count > max_count)
                max_count = current_count;
                
            // exit the loop if max_count is sufficiently large:
            if (max_count >= n - i)                 
                break;
                
            current_count = 1;
        }
        else                   // continuing the count of the current sequence:
        {
            current_count++;
            
            if (i == n-1 && current_count > max_count)
                max_count = current_count;
        }

        printf("\ta[%d]=%d; \tcurrent_count=%d; \tmax_count=%d\n",
               i, a[i], current_count, max_count);
    }
    return max_count;
}


\end{minted}	
	

	
	
	
	\paragraph{3. Write a BNF definition }\
	
	\rmfamily\
	
		The result of the program execution:
		
	\ttfamily
	\begin{lstlisting}[language=bash]

Array a:  1  1  1  2  3  3  5  6  6  6  6  7  9    
  a[0]=1; 	current_count=1; 	max_count=0
  a[1]=1; 	current_count=2; 	max_count=0
	a[2]=1; 	current_count=3; 	max_count=0
	a[3]=2; 	current_count=1; 	max_count=3
	a[4]=3; 	current_count=1; 	max_count=3
	a[5]=3; 	current_count=2; 	max_count=3
	a[6]=5; 	current_count=1; 	max_count=3
	a[7]=6; 	current_count=1; 	max_count=3
	a[8]=6; 	current_count=2; 	max_count=3
	a[9]=6; 	current_count=3; 	max_count=3
	a[10]=6; 	current_count=4; 	max_count=3
Max sequence length of array a = 4

Array b:    
Max sequence length of array b = 0

Array c:    12    
Max sequence length of array c = 1

Array d:    16    16    16    16    16    18    18    18    18    20    
	a[0]=16; 	current_count=1; 	max_count=0
	a[1]=16; 	current_count=2; 	max_count=0
	a[2]=16; 	current_count=3; 	max_count=0
	a[3]=16; 	current_count=4; 	max_count=0
	a[4]=16; 	current_count=5; 	max_count=0
Max sequence length of array d = 5

Array e:    0    0    
	a[0]=0; 	current_count=1; 	max_count=0
	a[1]=0; 	current_count=2; 	max_count=2
Max sequence length of array e = 2

Array f: 0    1    
	a[0]=0; 	current_count=1; 	max_count=0
Max sequence length of array f = 1

Array g:    1    2    3    4    5    5    5    5    
	a[0]=1; 	current_count=1; 	max_count=0
	a[1]=2; 	current_count=1; 	max_count=1
	a[2]=3; 	current_count=1; 	max_count=1
	a[3]=4; 	current_count=1; 	max_count=1
	a[4]=5; 	current_count=1; 	max_count=1
	a[5]=5; 	current_count=2; 	max_count=1
	a[6]=5; 	current_count=3; 	max_count=1
	a[7]=5; 	current_count=4; 	max_count=4
Max sequence length of array g = 4

Process finished with exit code 0

	\end{lstlisting}
	
	

	
	
	\rmfamily
	
	\paragraph{4. Write a BNF definition. }\
	
	\rmfamily\

	
		Following is an example :
		
	\ttfamily\

	

\end{document}