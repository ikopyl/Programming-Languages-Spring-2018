\documentclass{article}
\author{Ilya Kopyl}

\usepackage{amsmath, amssymb, amsthm}

\usepackage{enumitem}

\usepackage{listings}

\usepackage{minted}

\usepackage{syntax}

\usepackage[T1]{fontenc}			% https://tex.stackexchange.com/questions/2369/why-do-the-less-than-symbol-and-the-greater-than-symbol-appear-wrong-as

\usepackage{courier} % tt

\usepackage[english]{babel}
\usepackage[utf8]{inputenc}

\usepackage{pdfpages}

\usepackage{graphicx}
\graphicspath{ {images/} }

\usepackage{microtype}
\DisableLigatures[<]{encoding = T1}



\usepackage{listings}
\lstset{basicstyle=\ttfamily}



\begin{document}

	\setlength{\grammarparsep}{5pt plus 1pt minus 1pt} % increase separation between rules
%	\setlength{\grammarindent}{12em} % increase separation between LHS/RHS 
	\setlength{\grammarindent}{13em} % increase separation between LHS/RHS 
%	\setlength{\grammarindent}{5cm} 




	\begin{titlepage}
		\vspace*{\stretch{1.0}}
		\begin{center}
				\Large\textsc{CSc 600-01 (Section 1)}
				
				\Large\textbf{Homework 5 - Introduction to Ruby}\\

				\Large\textit{prepared by Ilya Kopyl}
				
		\end{center}	
		\vspace*{\stretch{2.0}}
	\end{titlepage}


	\title{\textsc{CSc 600 Homework 4 - Ruby Introduction}}	
	\maketitle
	
		\noindent \textit{Homework is prepared in LaTeX with TeXShop editor (under GNU GPL).}

	\rmfamily\




	\paragraph{1. Write a single Ruby demo program that illustrates the use of all main Ruby iterators (\(loop\), \(while\), \(until\), \(for\), \(upto\), \(downto\), \(times\), \(each\), \(map\), \(step\), \(collect\), \(select\), \(reject\)).}\

\paragraph{}\
1.1 loop

\begin{minted}[fontsize=\normalsize]{ruby}
# loop repeatedly executes the block of code
# In the example below I tried to emulate the look of vi text editor:
def use_loop
  line_number = 1
  loop do
    print "#{line_number}\t"
    line = gets
    break if line =~ /^\:q!|\:wq/      # exit on either :q! or :wq
    line_number += 1
  end
end
\end{minted}

Result of the code execution:

\begin{minted}[fontsize=\normalsize]{csh} 
$ irb -I . -r hw5_problem1.rb
irb(main):001:0> use_loop
1	Skepticism is a resting place for human reason
2	where it can reflect upon its dogmatic wanderings,
3	but it is no dwelling place for permanent settlement.
4	Simply to acquiesce in skepticism can never suffice
5	to overcome the restlessness of reason.:wq
=> nil
\end{minted}

\paragraph{}\
Depending on the existence and the location of the break statement inside the block, loop can be either a loop with exit at the top, with exit at the bottom, with exit in the middle, or with no exit at all, which would produce an infinite loop.

\paragraph{}\
If no block is given, an enumerator is returned instead:
\begin{minted}[fontsize=\normalsize]{bash} 
$ irb
irb(main):001:0> p loop
#<Enumerator: main:loop>
=> #<Enumerator: main:loop>
irb(main):002:0> puts loop
#<Enumerator:0x00007f813f09c140>
=> nil
\end{minted}




\paragraph{}\
\paragraph{}\


	\paragraph{2. Write Ruby recognizer methods \(limited?\) and \(sorted?\) that expand the Ruby class Array.}\ \newline
	
	The expression \texttt{array.limited?(amin, amax)} should return \(true\) if  \(amin  \leqslant  a[i]  \leqslant  amax\) \(\forall i\). \newline
	
	The expression \texttt{array.sorted?} should return the following: 
	\begin{itemize}
		\item \ \ \(0\) \quad  if the array is not sorted
		\item \(+1\) \quad if \(a[0] \leqslant a[1] \leqslant a[2] \leqslant ... \leqslant a[n]\) (non-decreasing order)
		\item \(-1\) \quad if \(a[0] \geqslant a[1] \geqslant a[2] \geqslant ... \geqslant a[n]\) (non-increasing order)
	\end{itemize}
	
	Show examples of the use of this method.

\paragraph{}\	
Source code of the program:

\begin{minted}[fontsize=\normalsize]{ruby} 

\end{minted}	

\paragraph{}\
	The result of the program execution:
	
\begin{minted}[fontsize=\normalsize]{bash} 

\end{minted}
	
\paragraph{}\
\paragraph{}\

\paragraph{3. Create a Ruby class \(triangle\) with initializer, accessors, and member functions for computing the \(perimeter\) and the \(area\) of arbitrary triangles. Also make a member function \(test\) that checks sides a, b, and c, and classifies the triangle as: }\

\begin{enumerate}[label=(\arabic*)]
	\item equilateral,
	\item isosceles,
	\item scalene,
	\item right,
	\item not a triangle.
\end{enumerate}

Right triangle can be either isosceles or scalene. Compute the perimeter and area only for valid triangles (verified by test). Show examples of the use of this class.

\paragraph{}\
\paragraph{}\
	The answer is listed on the page TBD.
\paragraph{}\
\paragraph{}\
Source code of the program:

\begin{minted}[fontsize=\normalsize]{ruby}

\end{minted}

\paragraph{}\
	The result of the program execution:
	
\begin{minted}[fontsize=\normalsize]{bash} 

\end{minted}


\paragraph{}\
\paragraph{}\



\paragraph{4. Create a Ruby class \(Sphere\). Each sphere is characterized by the instance variable radius. For this class create the initializer and the following methods: }\

	\begin{itemize}
		\item \(area\) \(-\) a method that returns the area of the sphere (\(a = 4r^2\pi\))
		\item \(volume\) \(-\) a method that returns the volume of the sphere (\(v = 4r^3\pi / 3\))
	\end{itemize}
	
	Create the class \(Ball\) that inherits properties from the class \(Sphere\) and adds a new instance variable \(color\). Then create the class \(MyBall\) that inherits properties from the class \(Ball\) and adds a new instance variable \(owner\). Write the method \(show\) that displays the instance variables of the class \(MyBall\). Show sample applications of the class \(MyBall\).

\paragraph{}\
	The answer is listed on the page TBD.
	
\paragraph{}\
\paragraph{}\
Source code of the program:

\begin{minted}[fontsize=\normalsize]{ruby}
\end{minted}

\paragraph{}\
\paragraph{}\
	Results of the program execution:
	
\begin{minted}[fontsize=\normalsize]{bash} 
\end{minted}

\paragraph{}\
\paragraph{}\

\end{document}